\documentclass[paper=a4, fontsize=11pt]{scrartcl} % A4 paper and 11pt font size
\usepackage[T1]{fontenc} % Use 8-bit encoding that has 256 glyphs
%\usepackage{fourier} % Use the Adobe Utopia font for the document - comment this line to return to the LaTeX default
\usepackage[english]{babel} % English language/hyphenation
\usepackage{amsmath,amsfonts,amsthm} % Math packages
\usepackage{advdate} %other dates than today
\usepackage{sectsty} % Allows customizing section commands
%\allsectionsfont{\centering \normalfont\scshape} % Make all sections centered, the default font and small caps
\usepackage{geometry}
\usepackage{listings} %allows to print code
\usepackage[per-mode=symbol]{siunitx} %units (hertz)
\usepackage{graphicx} %allows images
\usepackage{epstopdf} %enables eps in pdfs
\usepackage{float} %fixes images
\usepackage{caption} %allows changing captions
\usepackage{polynom} %polynomial long division
%hobbled together long division?
\usepackage[display]{texpower}
%\usepackage{titlesec}


\newcommand{\ldsym}{$\left.\mathstrut\right)$}% unbalanced )
\newlength{\ldwidth}

\newcommand{\longdivide}[2]% #1 = denominator, #2 = numerator
{\settowidth{\ldwidth}{\ldsym}
$#1\,\raisebox{1.5pt}{\ldsym}\hspace*{-.65\ldwidth}\overline{
\mathstrut\hspace*{.35\ldwidth}\ #2}$}

\usepackage{fancyhdr} % Custom headers and footers
\pagestyle{fancyplain} % Makes all pages in the document conform to the custom headers and footers
\fancyhead{} % No page header - if you want one, create it in the same way as the footers below
\fancyfoot[L]{} % Empty left footer
\fancyfoot[C]{} % Empty center footer
\fancyfoot[R]{\thepage} % Page numbering for right footer
\renewcommand{\headrulewidth}{0pt} % Remove header underlines
\renewcommand{\footrulewidth}{0pt} % Remove footer underlines
\setlength{\headheight}{0pt} % Customize the height of the header

\setlength\parindent{0pt} % Removes all indentation from paragraphs

\setlength{\parskip}{.5cm} % space between paragraphs

%margins
\geometry{a4paper, top=1in, left=1in, right=1in, bottom=1.4in, includehead, includefoot}

%Don't number sections
%\renewcommand{\thesection}

%Don't bold section headings
%\sectionfont{}
%\subsectionfont{\textnormal}

\lstset{basicstyle=\ttfamily\small,xleftmargin=-1cm,xrightmargin=-2cm,breaklines=true} %settings and font for code

\usepackage[utf8]{inputenc}
\DeclareFixedFont{\ttb}{T1}{txtt}{bx}{n}{9} % for bold
\DeclareFixedFont{\ttm}{T1}{txtt}{m}{n}{9}  % for normal
% Defining colors
\usepackage{color}
\definecolor{deepblue}{rgb}{0,0,0.5}
\definecolor{deepred}{rgb}{0.6,0,0}
\definecolor{deepgreen}{rgb}{0,0.5,0}


\usepackage{listings}

% Python style for highlighting
\newcommand\pythonstyle{\lstset{
language=Python,
backgroundcolor=\color{white},
basicstyle=\ttm,
otherkeywords={self},            
keywordstyle=\ttb\color{deepblue},
emph={MyClass,__init__},          
emphstyle=\ttb\color{deepred},    
stringstyle=\color{deepgreen},
commentstyle=\color{red},
frame=tb,                         
showstringspaces=false            
}}

% Python environment
\lstnewenvironment{python}[1][]
{
\pythonstyle
\lstset{#1}
}
{}
\renewcommand{\thesubsection}{\alph{subsection}} %Lettered subsections
%\renewcommand{\thesubsubsection}{\roman{subsubsection}} %lower case roman numeral subsubsections

\renewcommand{\labelenumi}{\arabic{enumi}.}


\newcommand{\horrule}[1]{\rule{\linewidth}{#1}} % Create horizontal rule command with 1 argument of height

\DeclareMathOperator*{\argmin}{arg\,min} %argmin

\title{
\normalfont \normalsize
\huge The Doodle Verse \\ % The assignment title
}
\date{}
\begin{document}
\maketitle

\section{Introduction}

Each civilization has their own set of constellations, groups of stars ascribed to a particular piece of cultural
memory. The ancient Greeks gave us most of our modern constellations: for example, Orion is a mythical
hunter, but the Lakota constellation of the Hand is roughly coincident (and consists of many of the same
stars). Could we take an arbitrary constellation, and then find a set of stars that fit it? In this lab, we’ll
explore how to take a set of about 1000 bright stars, and construct spectacular celestial displays of your own
design. Welcome to the Doodle Verse!

\section{Overview}

The problem can be broken up into two main tasks: (1) taking the input image and extracting the ideal feature points, and (2) finding the ideal set of stars whose formation matches this set of feature points.

For the second task, we start by framing it as an optimization problem. If $\mathbb{S}$ is our set of stars and $F$ is the matrix defined by our feature points, then we are looking for the subset $S\subset\mathbb{S}$ that minimizes the following: $||TF - S||_F$.  Here $T$ is a map from the Cartesian "feature space" to the space that our stars lie in that preserves the shape of our feature points (i.e. doesn't distort angles and relative distances).

The number of stars in our $\mathbb{S}$ set (which has been restricted to stars visible with the naked eye) is around 3000, so for $n$ feature points, there are around 3000 choose $n$ possibilities for $S$. The search space is too large for a brute force method.  We will proceed by breaking this problem into smaller problems to the point where a random search will yield sufficiently quick and accurate results.
\section{Warmup}

\subsection{Getting Feature Points}

Section about getting feature points.

\subsection{Mollweide Projection}
Next we want to begin our search to see if we can find a set of stars that fits these feature points. The first issue we encounter is that our extracted feature points have Cartesian coordinates in the $(x,y)-$plane and our star data set is in spherical coordinates. Here is a readout of the first few stars of our raw data set:
\begin{python}
	   RA         Dec        Mag  
	float64     float64    float64
	---------- ------------ -------
	0.02662528 -77.06529438    4.78
	0.03039927  -3.02747891    5.13
	0.03266433  -6.01397169    4.37
	0.03886504 -29.72044805    5.04
	0.06232565 -17.33597002    4.55
	0.07503398 -10.50949443    4.99
	0.08892938  -5.70783255    4.61
	0.13976888  29.09082805    2.07
	0.15280269  59.15021814    2.28
	0.15583908  -27.9879039    5.42
\end{python}
The "RA" or "right ascension" measures the distance from a central meridian and ranges from $0$ hours to $24$ hours (which corresponds to $0\si{\degree}-360\si{\degree}$). To convert to degrees we need to multiply this column by $15$.
The "Dec" or "declination" measures the distances above or below the equator and ranges from $-90\si{\degree}$ to $90\si{\degree}$.

To convert from spherical coordinates to Cartesian coordinates, we need to pick a center point and project the stars in the neighborhood of this point onto a plane.  If we attempt to project all or most of the stars, this will distort the stars far from our center point, which would be undesirable considering we are trying to match shapes.  This restricts our search to small neighborhoods at a time.

The Mollweide projection of a star with spherical coordinates $(\lambda,\phi)$ centered around ($\lambda_c,\phi_c$) can be obtained as follows:
$$
x = R\frac{2\sqrt{2}}{\pi}(\lambda-\lambda_c)cos(\theta)$$$$
y = R\sqrt{2}sin(\theta)
$$
Where $\theta$ is the angle defined by: $2\theta + sin(2\theta)=\pi sin(\phi-\phi_c)$.  Since $\theta$ is implicitly defined, we cannot solve for it directly. But the following iteration will converge to its value after a few iterations (it converges slow for points far from our center point, but that is not a concern for us).
$$
\theta_0 = \phi-\phi_c
$$$$
\theta_{k+1} = \theta_k + \frac{2\theta_k+sin(2\theta_k)-\pi(sin\phi-\phi_c)}{2+2cos(2\theta_k)}
$$

Here is a Python script which will compute the Mollweide projection for given spherical coordinates:
\begin{python}

def project(ra, dec, c_ra, c_dec):
	'''
	Finds the Mollweide projection coordinates (x,y) for the point (ra,dec) around 
	point (c_ra,c_dec).
	'''
	
	# Find theta
	theta_0 = dec - c_dec
	epsilon = 10**-6
	error = 1+epsilon
	
	while error > epsilon:
	    m = (2*theta_0+np.sin(2*theta_0)-np.pi*np.sin(dec - c_dec))/(2+2*np.cos(2*theta_0))
	    theta_1 = theta_0 - m
	    error = np.abs(theta_1 - theta_0)
	    theta_0 = theta_1
	
	# Compute (x,y) coordinates
	x = 2*np.sqrt(2)*(ra-c_ra)*np.cos(theta_0)/np.pi
	y = np.sqrt(2)*np.sin(theta_0)
	
	return [x,y]
\end{python}

Use this script to convert the points found in (something) to (x,y) coordinates centered at the point (???).

\subsection{Preliminary Search: A Clustering Approach}
We wish to determine if there is possible good match for our feature points in the set obtained from the previous section.  Since we are only trying to match the shape and not the scale of our feature points, one approach is to find a set of stars that forms the same set of interior angles.

In essence, this is a clustering problem. First, let's simplify the problem by considering the case where our feature points consist of just 3 points: $(p_1,p_2,p_3)$. The shape of these 3 points is uniquely determined by their interior angles (i.e. the angles of the triangle they form), call them $(a_1,a_2,a_3)$. If we think of every set of 3 points in our search space as a specified by their interior angles (we can always order the angles in ascending order to uniquely determine the set's "angle representation"), we are basically looking for the 3-point sets that lie in the same cluster as our feature points. This cluster would be our "possible match" cluster.

Question: Form this as a clustering problem $X = DW$. What do $X,D,W$ look like? Say our search set contains $50$ stars. If D is $n\times k$, what is $n$? What would be the effect of choosing smaller or larger values of $k$?

Unfortunately, even with our substantially restricted search space ($30-50$ stars), the number of possible size 3 subsets is still quite large (on the order of $10^4$), and the clustering algorithms we know require you to iterate through every point. Fortunately, it turns out that solving this cluster problem in its entirety is not necessary. We are not only looking for the \textit{best fit} sets of stars, we are looking for the best fit \textit{and brightest} sets of stars. By sorting the stars by magnitude, we can prioritize clustering the brightest stars so we find the brightest, best fits.

So if we are looking for $m$ possible matches in our subset, we can start with a much smaller subset of brightest stars and keep adding the next-brightest star until the cluster that contains our feature points has $m$ members.

Unfortunately, this approach doesn't scale well when our feature set is more than 3 points (most notably, the search space becomes way too large). However, matching 3 feature points to 3 stars does uniquely determine a transformation from the $(x,y)-$plane that the features lie in to the $(x,y)-$plane that the stars lie in. So rather than finding a match for the remaining points using this clustering approach, we can simply check to see if there are stars in the appropriate places to form the rest of our shape. To do this, we need to find the transformation from the $(x,y)-$coordinates of our features to the $(x,y)-$coordinates of our stars. To find this transformation, we must solve what is called the Procrustes problem.

\subsection{The Orthogonal Procrustes Problem}
Here we address the problem of finding the ideal transformation $T$ between our set of feature points $F$ and a predetermined subset of stars $S\subset \mathbb{S}$. Here we are assuming $F$ and $S$ are the same size $(2\times k)$ and represent their respective collections of points in Cartesian coordinates (each column is a point).  Finding this transformation will be necessary to evaluate the fitness of a possible match.

Other than minimizing the distance between points, the primary goal is preserving the shape of $F$. For now, let's also constrain the problem to where $S$ and $F$ are normalized, so we don't need to change the scale of $F$: our transformation will preserve distances and angles between points (and thus the shape). This simplifies the problem because it constrains $T$ to be orthogonal. Also, let's assume that both sets of points are centered around the origin (i.e. their mean is $\begin{bmatrix}
0\\
0
\end{bmatrix}$). Basically, we are looking for the best rotation of $F$ that minimizes the distance between its points and the points in $S$. Precisely, we are looking for the orthogonal $T$ that minimizes: 
$$||TF-S||^2_F = \langle TF-S, TF-S\rangle_F$$
Where $tr(AB^T) = \langle A,B\rangle_F = \langle B,A\rangle_F = tr(BA^T)$.

Let $U\Sigma V^T$ be the SVD of the matrix $SF^T$. Show that the orthogonal $T$ that minimizes\\ $||TF-S||^2_F$ is equal to $UV^T$.

Hint: $T,U$, and $V$ are orthogonal, and multiplying a vector by an orthogonal matrix doesn't change its Frobenius norm.

Here is the Python code that solves the orthogonal Procrustes problem:

\begin{python}
def Procrustes(F, S):
	'''
	Given F, S (same size, unit norm, and centered around the origin), finds the
	ideal transformation that maps the points of F to best fit the points of S.
	That is, ||TF-S|| is minimized.
	'''
	
	[U,s,VT] = np.linalg.svd(np.dot(S,F.T))
	
	T = np.dot(U,VT)
	
	return T
\end{python}

\subsection{Evaluating the Match}

Now we have sufficient machinery to evaluate our possible match.  We know we have a decent match for three of our feature points, and we know how to transform these feature points so they best align with the match. We just have to determine if there is decent match for the remaining feature points. To do this, we apply the same transformation to these feature points, and search for bright stars nearby the transformed points. In the simplest case, we could just pick the closest star to each transformed point.

Now we have our set of transformed feature points $[f_1,\ldots,f_n]=\bar{F}$, our set of stars $[s_1,\ldots,s_n] = \bar{S}$, with $[m_1,\ldots,m_n] = \bar{M}$ being the set of magnitudes for the stars in $\bar{S}$.

Suppose we had $k$ possible matches $\{\bar{S}_i,\bar{M}_i\}$, and we want to determine the \textit{best} one. Set up a cost function that we are trying to minimize.


\section{Lab}

\begin{enumerate}
  \item Feature detection part
  \item
  Generalize the orthogonal procrustes problem so that it will be useful for our case. That is, given the $(2\times k)$ matrix of feature points $F$ and a  $(2\times k)$ matrix of star points $S$ (not necessarily unit norm or centered around the origin), find ideal transformation $T$ such that $||T(F)-S||_F$ is minimized.  Note that it is easy to reduce to the previous case by computing the following:
  $$
  \hat{F} = \frac{F}{||F||_F} - mean(F)\quad\textrm{ and }\quad\hat{S} = \frac{S}{||S||_F} - mean(S)
  $$
  and solving the problem for $\hat{F}$ and $\hat{S}$. Modify the previous Procrustes problem code to solve the general Procrustes problem.
  \item Question 2...
  \item Question 3....
\end{enumerate}



\end{document}
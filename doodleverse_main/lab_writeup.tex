\documentclass[paper=a4, fontsize=11pt]{scrartcl} % A4 paper and 11pt font size
\usepackage[T1]{fontenc} % Use 8-bit encoding that has 256 glyphs
%\usepackage{fourier} % Use the Adobe Utopia font for the document - comment this line to return to the LaTeX default
\usepackage[english]{babel} % English language/hyphenation
\usepackage{amsmath,amsfonts,amsthm} % Math packages
\usepackage{advdate} %other dates than today
\usepackage{sectsty} % Allows customizing section commands
%\allsectionsfont{\centering \normalfont\scshape} % Make all sections centered, the default font and small caps
\usepackage{geometry}
\usepackage{listings} %allows to print code
\usepackage[per-mode=symbol]{siunitx} %units (hertz)
\usepackage{graphicx} %allows images
\usepackage{epstopdf} %enables eps in pdfs
\usepackage{float} %fixes images
\usepackage{caption} %allows changing captions
\usepackage{polynom} %polynomial long division
%hobbled together long division?
\usepackage[display]{texpower}

\newcommand{\ldsym}{$\left.\mathstrut\right)$}% unbalanced )
\newlength{\ldwidth}

\newcommand{\longdivide}[2]% #1 = denominator, #2 = numerator
{\settowidth{\ldwidth}{\ldsym}
$#1\,\raisebox{1.5pt}{\ldsym}\hspace*{-.65\ldwidth}\overline{
\mathstrut\hspace*{.35\ldwidth}\ #2}$}

\usepackage{fancyhdr} % Custom headers and footers
\pagestyle{fancyplain} % Makes all pages in the document conform to the custom headers and footers
\fancyhead{} % No page header - if you want one, create it in the same way as the footers below
\fancyfoot[L]{} % Empty left footer
\fancyfoot[C]{} % Empty center footer
\fancyfoot[R]{\thepage} % Page numbering for right footer
\renewcommand{\headrulewidth}{0pt} % Remove header underlines
\renewcommand{\footrulewidth}{0pt} % Remove footer underlines
\setlength{\headheight}{0pt} % Customize the height of the header

%\setlength\parindent{0pt} % Removes all indentation from paragraphs

\setlength{\parskip}{.5cm} % space between paragraphs

%margins
\geometry{a4paper, top=1in, left=1in, right=1in, bottom=1.4in, includehead, includefoot}

%Don't number sections
%\renewcommand{\thesection}

%Don't bold section headings
%\sectionfont{}
%\subsectionfont{\textnormal}

\lstset{basicstyle=\ttfamily\small,xleftmargin=-1cm,xrightmargin=-2cm,breaklines=true} %settings and font for code

\renewcommand{\thesubsection}{\alph{subsection}} %Lettered subsections
\renewcommand{\thesubsubsection}{\roman{subsubsection}} %lower case roman numeral subsubsections
\newcommand{\horrule}[1]{\rule{\linewidth}{#1}} % Create horizontal rule command with 1 argument of height

\DeclareMathOperator*{\argmin}{arg\,min} %argmin

\title{
\normalfont \normalsize
\huge The Doodle Verse \\ % The assignment title
}
\date{}
\begin{document}
\maketitle

\section{Introduction}

\section{Overview}

\section{Warmup}

\subsection{Mollweide Projection}
One issue is that our extracted feature points will have Cartesian coordinates and our star data set is in spherical coordinates. Here is a readout of the first 10 stars of our data set:
\begin{lstlisting}
    RA         Dec        Mag  
 float64     float64    float64
---------- ------------ -------
0.02662528 -77.06529438    4.78
0.03039927  -3.02747891    5.13
0.03266433  -6.01397169    4.37
0.03886504 -29.72044805    5.04
0.06232565 -17.33597002    4.55
0.07503398 -10.50949443    4.99
0.08892938  -5.70783255    4.61
0.13976888  29.09082805    2.07
0.15280269  59.15021814    2.28
0.15583908  -27.9879039    5.42
\end{lstlisting}
The "RA" or measures the distance from a central meridian and ranges from $0\si{\degree}$ to $360\si{\degree}$.
The "Dec" or "declination" measures the distances above or below the equator and ranges from $-90\si{\degree}$ to $90\si{\degree}$.

To convert from spherical coordinates to Cartesian coordinates, we need to pick a center point and project the stars in the neighborhood of this point onto a plane.  If we attempt to project all or most of the stars, this will distort the stars far from our center point, which would be undesirable considering we are trying to match shapes.  This restricts our search to small neighborhoods at a time.

The Mollweide projection of a star with spherical coordinates $(\lambda,\phi)$ centered around ($\lambda_c,\phi_c$) can be obtained as follows:
$$
x = R\frac{2\sqrt{2}}{\pi}(\lambda-\lambda_c)cos(\theta)$$$$
y = R\sqrt{2}sin(\theta)
$$
Where $\theta$ is the angle defined by: $2\theta + sin(2\theta)=\pi sin(\phi-\phi_c)$.  Since $\theta$ is implicitly defined, we cannot solve for it directly. But the following iteration will converge to its value quickly (it converges slow for points far from our center point, but that is not a concern for us).
$$
\theta_0 = \phi-\phi_c
$$$$
\theta_{k+1} = \theta_k + \frac{2\theta_k+sin(2\theta_k)-\pi(sin\phi-\phi_c)}{2+2cos(2\theta_k)}
$$

Right a script that does this?


\subsection{Procrustes}
Here we address the problem of finding the ideal transformation $T$ between our set of feature points $F$ and a predetermined subset of stars $S\subset \mathbb{S}$. Here we are assuming $F$ and $S$ are the same size $(k\times2)$ and represent their respective collections of points in Cartesian coordinates. We want to preserve the shape of $F$, and for now let's assume we also don't want to change the scale of $F$: our transformation will preserve distances and angles between points (and thus the shape). This simplifies the problem because it constrains $T$ to be orthogonal. Precisely, we are looking for the orthogonal $T$ that minimizes $||FT-S||_F$.

\section{Lab}





\end{document}
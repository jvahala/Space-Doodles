\documentclass{article}
\usepackage{geometry}
\usepackage{graphicx} %allows images
\usepackage{epstopdf} %enables eps in pdfs
\usepackage{float} %fixes images
\usepackage{siunitx} %units (hertz)
\usepackage{amsmath}
\usepackage{bibentry} %enables displaying of paper names
\title{ECE 532 Project Propsal}
\date{}
\author{}
\geometry{margin=.5in}

\begin{document}

%define bibliography (but do not display it)
\bibliographystyle{plain}
\nobibliography{projectProposal}

\maketitle

\section{Tentative Title: The DongleVerse\textsuperscript{\textregistered}}

\section{Team Members}
Devin Conathan, Josh Vahala

\section{Brief Overview of Topic and Motivation}

\section{Core Concepts}

\section{Related Papers, Datasets, or Resources}
For data bases.... there's that star data base. I'm finding out how to "cite" that.
There are plenty of papers dealing with a similar problem - taking an image of the sky or star map and finding known constellations, which is useful for satellite orientation and navigation:
\\\\
\bibentry{Renken}
\\\\
\bibentry{Krips2011}
\\\\
\bibentry{Ji}

\end{document} 